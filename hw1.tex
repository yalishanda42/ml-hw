\documentclass{article}
\usepackage{amsmath}
\usepackage[utf8]{inputenc}
\usepackage[bulgarian]{babel}

\title{Упражнение 1}
\author{Александър Игнатов}
\date{\today}


\begin{document}

\maketitle

\section*{Задача 1}

За тази задача сме избрали да представим хипотезите като
конюнкции на ограниченията на атрибутите.
Атрибутите в нашата задача са ,,небе``, ,,въздух``,
,,влажност``, ,,вятър``, ,,вода`` и ,,прогноза``,
всяко от тях с по 2 възможни стойности
(с изключение на атрибутът ,,небе``,
който има три възможни стойности),
т.е. всички възможни комбинации на стойностите на
атрибутите са \( 3 \times 2 \times 2 \times 2 \times 2 \times 2 = 96 \), което е и размерът на пространството от примери.

Размерът на пространството от хипотези е по-голям, понеже към всяка възможна стойност на един атрибут добавяме още две ограничения: ,,\(\emptyset\)`` (нито една стойност не е допустима) и ,,?`` (всички стойности са допустими).
Така получаваме \( 5 \times 4 \times 4 \times 4 \times 4 \times 4 = 5120 \) синтактически различни хипотези.

Голяма част от тях обаче съвпадат, понеже \( a_1 \land a_2 \land ... \land a_n = 0 \), ако поне едно \( a_i = \emptyset \) (\( i \in [1, n] \)).
Така всички хипотези, които съдържат конюнкция на \(\emptyset\) в себе си са на практика една и съща такава,
следователно броят на семантично различните хипотези се свежда до \( 4 \times 3 \times 3 \times 3 \times 3 \times 3 + 1 = 973 \).

При добавяне на нов атрибут с 3 възможни стойности, бройката на възможните примери се умножава с 3 и нараства до \( 3 \times 2 \times 2 \times 2 \times 2 \times 2 \times 3 = 288 \),
а броят на възможните семантично-различни хипотези -- \( 4 \times 3 \times 3 \times 3 \times 3 \times 3 \times 4 + 1 = 3889 \).

В общия случай, ако имаме \( X_0 \) на брой примери и \( H_0 \) на брой хипотези, при добавяне на нов атрибут с \( k \) на брой възможни стойности,
то новият брой примери и хипотези става респективно:

\begin{align*}
    X_1 &= k X_0 \\
    H_1 &= k (H_0 - 1) + 1 \\
\end{align*}

\end{document}

